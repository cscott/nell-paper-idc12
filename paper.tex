% THIS IS SIGPROC-SP.TEX - VERSION 3.1
% WORKS WITH V3.2SP OF ACM_PROC_ARTICLE-SP.CLS
%
% REMEMBER HOWEVER: After having produced the .bbl file,
% and prior to final submission,
% you need to 'insert'  your .bbl file into your source .tex file so as to provide
% ONE 'self-contained' source file.
\documentclass{acm_proc_article-sp}

\begin{document}

\title{Growing up with Nell: Narrative Interfaces for Literacy}
%\subtitle{Narrative Interfaces for Literacy}
%
% You need the command \numberofauthors to handle the 'placement
% and alignment' of the authors beneath the title.
%
% For aesthetic reasons, we recommend 'three authors at a time'
% i.e. three 'name/affiliation blocks' be placed beneath the title.
%
% NOTE: You are NOT restricted in how many 'rows' of
% "name/affiliations" may appear. We just ask that you restrict
% the number of 'columns' to three.
%
% Because of the available 'opening page real-estate'
% we ask you to refrain from putting more than six authors
% (two rows with three columns) beneath the article title.
% More than six makes the first-page appear very cluttered indeed.
%
% Use the \alignauthor commands to handle the names
% and affiliations for an 'aesthetic maximum' of six authors.
% Add names, affiliations, addresses for
% the seventh etc. author(s) as the argument for the
% \additionalauthors command.
% These 'additional authors' will be output/set for you
% without further effort on your part as the last section in
% the body of your article BEFORE References or any Appendices.

\numberofauthors{1}
\author{
% You can go ahead and credit any number of authors here,
% e.g. one 'row of three' or two rows (consisting of one row of three
% and a second row of one, two or three).
%
% The command \alignauthor (no curly braces needed) should
% precede each author name, affiliation/snail-mail address and
% e-mail address. Additionally, tag each line of
% affiliation/address with \affaddr, and tag the
% e-mail address with \email.
%
% 1st. author
\alignauthor
C. Scott Ananian\\
\affaddr{One Laptop Per Child Foundation}\\
\affaddr{222 Third Street}\\
\affaddr{Cambridge, MA 02142}\\
\email{cscott@laptop.org}
%\and  % use '\and' if you need 'another row' of author names
}
\date{12 March 2012}
% Just remember to make sure that the TOTAL number of authors
% is the number that will appear on the first page PLUS the
% number that will appear in the \additionalauthors section.

\maketitle

\begin{abstract}
We present Project Nell,%
\footnote{We were inspired by the ``Young Lady's Illustrated
Primer'' in the Neal Stephenson novel \textit{The Diamond Age}.
Nell takes its name from that novel's protagonist, but could also be read as an
acronym for ``Narrative Environment for Learning Learning'' (a nod to
Seymour Papert).}
tablet software to teach reading and writing to
children far from educational infrastructure.  Nell is a modular
narrative system with ``a low floor and no ceilings'': the design
scales from teaching letter shapes to programming the system itself.

Nell is the successor to the Sugar educational software developed by
One Laptop per Child for their XO-1 laptop.  Nell targets the XO-3
tablet, a sub-\$100 solar-powered tablet for the developing world.
Experience with Sugar drove design choices for Nell.
\end{abstract}

% A category with the (minimum) three required fields
\category{K.3.1}{Computers and Education}{Computer
  Uses in Education}[computer-managed instruction]
\category{I.2.7}{Artificial Intelligence}{Natural Language
  Processing}[narrative interfaces]
\category{H.5.2}{Information Interfaces and Presentation}{User
  Interfaces}[narrative interfaces]
\terms{Design, Human Factors}

\keywords{Narrative interfaces, tablet computing, education, Nell}

\section{Introduction}
\section{The {\secit Body} of The Paper}
\subsection{Type Changes and {\subsecit Special} Characters}
\subsection{Math Equations}
\subsection{Citations}
Citations to articles \cite{bowman:reasoning, clark:pct, braams:babel, herlihy:methodology},
conference
proceedings \cite{clark:pct} or books \cite{salas:calculus, Lamport:LaTeX} listed
in the Bibliography section of your
article will occur throughout the text of your article.

This article shows only the plainest form
of the citation command, using \texttt{{\char'134}cite}.
This is what is stipulated in the SIGS style specifications.
No other citation format is endorsed.

\section{Conclusions}
This paragraph will end the body of this sample document.
Remember that you might still have Acknowledgments or
Appendices; brief samples of these
follow.  There is still the Bibliography to deal with; and
we will make a disclaimer about that here: with the exception
of the reference to the \LaTeX\ book, the citations in
this paper are to articles which have nothing to
do with the present subject and are used as
examples only.
%\end{document}  % This is where a 'short' article might terminate

%
% The following two commands are all you need in the
% initial runs of your .tex file to
% produce the bibliography for the citations in your paper.
\bibliographystyle{abbrv}
\bibliography{sigproc}  % sigproc.bib is the name of the Bibliography in this case
% You must have a proper ".bib" file
%  and remember to run:
% latex bibtex latex latex
% to resolve all references
%
% ACM needs 'a single self-contained file'!
%
\balancecolumns
% That's all folks!
\end{document}
