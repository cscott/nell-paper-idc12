% THIS IS SIGPROC-SP.TEX - VERSION 3.1
% WORKS WITH V3.2SP OF ACM_PROC_ARTICLE-SP.CLS
%
% REMEMBER HOWEVER: After having produced the .bbl file,
% and prior to final submission,
% you need to 'insert'  your .bbl file into your source .tex file so as to provide
% ONE 'self-contained' source file.
\documentclass{acm_proc_article-sp}

\begin{document}

\title{Growing up with Nell: A Narrative Interface for Literacy}
%\subtitle{Narrative Interfaces for Literacy}
%
% You need the command \numberofauthors to handle the 'placement
% and alignment' of the authors beneath the title.
%
% For aesthetic reasons, we recommend 'three authors at a time'
% i.e. three 'name/affiliation blocks' be placed beneath the title.
%
% NOTE: You are NOT restricted in how many 'rows' of
% "name/affiliations" may appear. We just ask that you restrict
% the number of 'columns' to three.
%
% Because of the available 'opening page real-estate'
% we ask you to refrain from putting more than six authors
% (two rows with three columns) beneath the article title.
% More than six makes the first-page appear very cluttered indeed.
%
% Use the \alignauthor commands to handle the names
% and affiliations for an 'aesthetic maximum' of six authors.
% Add names, affiliations, addresses for
% the seventh etc. author(s) as the argument for the
% \additionalauthors command.
% These 'additional authors' will be output/set for you
% without further effort on your part as the last section in
% the body of your article BEFORE References or any Appendices.

\numberofauthors{1}
\author{
% You can go ahead and credit any number of authors here,
% e.g. one 'row of three' or two rows (consisting of one row of three
% and a second row of one, two or three).
%
% The command \alignauthor (no curly braces needed) should
% precede each author name, affiliation/snail-mail address and
% e-mail address. Additionally, tag each line of
% affiliation/address with \affaddr, and tag the
% e-mail address with \email.
%
% 1st. author
\alignauthor
C. Scott Ananian\\
\affaddr{One Laptop Per Child Foundation}\\
\affaddr{222 Third Street}\\
\affaddr{Cambridge, MA 02142}\\
\email{cscott@laptop.org}
%\and  % use '\and' if you need 'another row' of author names
}
\date{12 March 2012}
% Just remember to make sure that the TOTAL number of authors
% is the number that will appear on the first page PLUS the
% number that will appear in the \additionalauthors section.

\maketitle

\begin{abstract}
Project Nell is tablet software to teach reading and writing to
children far from educational infrastructure.  Nell is a modular
narrative system with ``a low floor and no ceilings'': the design
scales from teaching letter shapes to programming the system itself.

Nell is the successor to the Sugar educational software developed by
One Laptop per Child for their XO-1 laptop.  Nell targets the XO-3
tablet, a sub-\$100 solar-powered tablet for the developing world.
%Experience with Sugar drove design choices for Nell.

\textbf{XXX REWRITE THE ABSTRACT XXX}
\end{abstract}

% A category with the (minimum) three required fields
\category{K.3.1}{Computers and Education}{Computer
  Uses in Education}[computer-managed instruction]
\category{I.2.7}{Artificial Intelligence}{Natural Language
  Processing}[narrative interfaces]
\category{H.5.2}{Information Interfaces and Presentation}{User
  Interfaces}[narrative interfaces]
\terms{Design, Human Factors}

\keywords{Narrative interfaces, tablet computing, education, Nell}

\section{A Lesson from Nell}

% tell a nell story up front
Miles from the nearest school, a young girl in Ethiopia turns on a
tablet computer for the first time.  The solar-powered machine begins
talking to her immediately: \textit{``Hello!  Would you like to hear a story?''}

She listens intently to a story about a princess named Nell.  Later, when
the girl has learned a little more, she will give the princess her own
name, ``Rahel,'' and change all the colors, but for now the green
book draws pictures for her and asks her to trace shapes on the screen.

\textit{``R is for Raven.  Can you trace the R?''}

As she traces the R, it comes to life and gallops across the screen.
\textit{``Run starts with R.  Roger the R Runs across the Red Rug.  He has a
dog named Rover.''}  Rover barks: \textit{``Ruff! Ruff!''}  The Princess asks,
\textit{``Can you find something Red?''} and Rahel uses the camera to
photograph her red boots.

\textit{``Very good!  You are a smart girl!''}

As Rahel grows, the book asks her to trace not just letters, but whole
words.  The book's responses are written on the screen as it speaks
them, and eventually she doesn't need to leave the sound on all the
time.  Soon Rahel can write complete sentences in her special book,
and the Princess will respond sometimes.  New stories teach her about
music (by playing certain tunes she unlocks a dungeon door) and
programming with blocks (a not-very-bright turtle needs Princess Rahel's
help to draw different shapes).  Rahel writes her own stories,
which she shares with her friends.

An older Rahel learns that the block language she used to talk with
the turtle is also used to write all the software running inside
her special book.  The stories Rahel writes can now include new
functionality that make her book even more exciting for the kids
just now opening their green meta books.

\section{Key Ideas}
The interaction design of the Nell system described above is inspired
by the ``Young Lady's Illustrated Primer'' in the Neal Stephenson
novel \textit{The Diamond Age}, from whose protagonist Nell takes its
name.\footnote{In a nod to Seymour Papert, Nell can also be read as an
acronym for ``Narrative Environment for Learning Learning.''}
Nell's design embodies four key ideas: it is a \textbf{Narrative}
interface using \textbf{Direct Interaction} which \textbf{Grows} with,
and is \textbf{Personalized} for, the child.

We believe the combination of these four key ideas provides a novel
experience which addresses some of the shortcomings of earlier
learning environments for children.  In this section we will discuss
our motivating concepts, and in the next we will describe how
they are implemented in Nell.

%Section~\ref{sec:related} discusses related work.
%The final section will draw conclusions and describe future work.

\subsection{Narrative Interface}

Experience with our previous educational software
system~\cite{flores:uruguay,idb:peru} has shown gratifying uptake and
use by children.  But we are often disappointed that some of our
favorite pedagogical tools included in the systems were not being
widely used.  Enthusiastic teachers would spark intensive use in
certain classrooms, but children had difficulty finding their way into
the material without strong guidance.

Children (like all humans) are hard-wired for
stories~\cite{boyd:stories}.  Nell uses a
\textit{Narrative Interface}~\cite{bizzocchi:narrative,don:narrative,laurel:computers} 
to guide the child user through the pedagogical capabilities of the system.

% excerpt of story model
% not always a story -- kid can draw on their own and return to the
% story; Nell will observe the free play and offer suggestions.
\subsection{Personal}
% customization, renaming characters
\subsection{Direct interaction}
% handwriting as UI?
% motor skills
% photo of XO-3?
\subsection{Growable (``no ceiling'')}
% it has a ``low floor and no ceiling'' to allow a child to master more
% and more of the system as she grows.

% modular -- the story expands.
% excerpt of TurtleScript
% text read aloud initially

\section{Technology}
\subsection{Rule-based story modules}
% rete algorithm, cite Riedl, etc.
% programmable conflict resolution
\subsection{Simple dialog with AIML}
% eliza interactions, show fragment
\subsection{Web technologies}
% Javascript/HTML5
% offline caching
% turtles (almost) all the way down (turtlescript again)

\section{Related work}\label{sec:related}
% Nell focused on Literacy
% olpc context here?
\cite{chang:tinkrbook}
%context: english literacy, far from teachers & schools
\subsection{The Sugar Learning Environment}
% Sugar
%% customization
%% path through the educational material?
\subsection{Narrative computing}
% Reidl
\subsection{Storytelling}
% TinkRBooks

\section{Conclusions}

\section{Acknowledgments}
Nell's design evolved from experience with the Sugar system,
architected by Walter Bender, and through stimulating discussions with
Chris Ball and Angela Chang.  Michael Stone also contributed to an
early implementation.

%
% The following two commands are all you need in the
% initial runs of your .tex file to
% produce the bibliography for the citations in your paper.
\bibliographystyle{abbrv}
\bibliography{paper}  % paper.bib is the name of the Bibliography in this case
% You must have a proper ".bib" file
%  and remember to run:
% latex bibtex latex latex
% to resolve all references
%
% ACM needs 'a single self-contained file'!
%
\balancecolumns
% That's all folks!
\end{document}
